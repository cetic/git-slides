\documentclass[a4paper,10pt,landscape]{article}
\usepackage{multicol}
\usepackage{calc}
\usepackage{ifthen}
\usepackage[landscape]{geometry}
\usepackage{hyperref}
\usepackage[utf8]{inputenc}

\usepackage[framemethod=TikZ]{mdframed}
\mdfdefinestyle{mystyle}{linewidth=1pt, innerleftmargin=5pt, innerrightmargin=5pt, linecolor=black, backgroundcolor=gray!20}
\mdfdefinestyle{tip}{linewidth=1pt, innerleftmargin=5pt, innerrightmargin=5pt, linecolor=black, roundcorner=10pt, backgroundcolor=yellow!20}



% To make this come out properly in landscape mode, do one of the following
% 1.
%  pdflatex latexsheet.tex
%
% 2.
%  latex latexsheet.tex
%  dvips -P pdf  -t landscape latexsheet.dvi
%  ps2pdf latexsheet.ps


% If you're reading this, be prepared for confusion.  Making this was
% a learning experience for me, and it shows.  Much of the placement
% was hacked in; if you make it better, let me know...


% 2008-04
% Changed page margin code to use the geometry package. Also added code for
% conditional page margins, depending on paper size. Thanks to Uwe Ziegenhagen
% for the suggestions.

% 2006-08
% Made changes based on suggestions from Gene Cooperman. <gene at ccs.neu.edu>


% To Do:
% \listoffigures \listoftables
% \setcounter{secnumdepth}{0}


% This sets page margins to .5 inch if using letter paper, and to 1cm
% if using A4 paper. (This probably isn't strictly necessary.)
% If using another size paper, use default 1cm margins.
\ifthenelse{\lengthtest { \paperwidth = 11in}}
	{ \geometry{top=.5in,left=.5in,right=.5in,bottom=.5in} }
	{\ifthenelse{ \lengthtest{ \paperwidth = 297mm}}
		{\geometry{top=1cm,left=1cm,right=1cm,bottom=1cm} }
		{\geometry{top=1cm,left=1cm,right=1cm,bottom=1cm} }
	}

% Turn off header and footer
\pagestyle{empty}
 

% Redefine section commands to use less space
\makeatletter
\renewcommand{\section}{\@startsection{section}{1}{0mm}%
                                {-1ex plus -.5ex minus -.2ex}%
                                {0.5ex plus .2ex}%x
                                {\normalfont\large\bfseries}}
\renewcommand{\subsection}{\@startsection{subsection}{2}{0mm}%
                                {-1explus -.5ex minus -.2ex}%
                                {0.5ex plus .2ex}%
                                {\normalfont\normalsize\bfseries}}
\renewcommand{\subsubsection}{\@startsection{subsubsection}{3}{0mm}%
                                {-1ex plus -.5ex minus -.2ex}%
                                {1ex plus .2ex}%
                                {\normalfont\small\bfseries}}
\makeatother

% Define BibTeX command
\def\BibTeX{{\rm B\kern-.05em{\sc i\kern-.025em b}\kern-.08em
    T\kern-.1667em\lower.7ex\hbox{E}\kern-.125emX}}

% Don't print section numbers
\setcounter{secnumdepth}{0}


\setlength{\parindent}{0pt}
\setlength{\parskip}{0pt plus 0.5ex}


% -----------------------------------------------------------------------

\begin{document}

\raggedright
\footnotesize
\begin{multicols}{3}

\newlength{\MyLen}

% multicol parameters
% These lengths are set only within the two main columns
%\setlength{\columnseprule}{0.25pt}
\setlength{\premulticols}{1pt}
\setlength{\postmulticols}{1pt}
\setlength{\multicolsep}{1pt}
\setlength{\columnsep}{2pt}

\begin{center}
     \Large{\textbf{Git Cheat Sheet v1.0}} \\
     \large{by Sébastien Dawans}
\end{center}

\section{Terminology}
\begin{mdframed}[style=mystyle]
\settowidth{\MyLen}{\texttt{organization.}}
\begin{tabular}{@{}p{\the\MyLen}@{}p{\linewidth-\the\MyLen}@{}}
\verb!repository!         &  Contains the full history of a project.  \\
\verb!remote!              &  Short for remote repository.  \\
\verb!origin!                &  Default name given to a remote repository.  \\
\verb!working tree!     &  The place where a specific version of the project is made available. \\
\verb!branch!              &  A movable pointer to a \texttt{[commit]}. \\
\verb!tag!                   &  An immutable pointer to a \texttt{[commit]}. \\
\verb!master!              &  The default branch name. \\
\verb!blob!                  &  A git object containing the compressed contents of a versioned file. \\
\verb!tree!                  &  Represents a directory. Contains \texttt{blobs} and other \texttt{trees.} \\
\verb!commit!              &  A particular version of the project. Points to a \texttt{tree} and meta-data (author, date, message...). \\
\verb!author!              & Person who wrote a particular changeset. \\
\verb!committer!         & Person who applied the changeset. \\
\verb!the index!          & Also called Staging Area, stores changes to be committed. \\
\verb!to stage!            & To place a change in the git index. \\
\verb!to unstage!        & To remove a change from the staging area, leaving it for a future commit. \\
\end{tabular}
\end{mdframed}


\subsection{Referencing}
\begin{mdframed}[style=mystyle]
\settowidth{\MyLen}{\texttt{organization.}}
\begin{tabular}{@{}p{\the\MyLen}@{}p{\linewidth-\the\MyLen}@{}}
\verb!HEAD!                & A pointer to the current branch. \\
\verb!commit-ish!      & An identifier that ultimately leads to a \texttt{commit} object. Shortened as \texttt{<ref>} \\
\verb!tree-ish!         & An identifier that ultimately leads to a \texttt{tree} object. \\
\texttt{<ref>\string~\string{n\string}}        & n-th generation ancestor of \texttt{<rev>}. Example: \texttt{HEAD\string~2} \\
\texttt{<ref>\string^\string{n\string}}        & n-th parent of \texttt{<rev>}. Example: \texttt{HEAD\string^2} \\
\end{tabular}
\end{mdframed}

\section{Git Commands}

\subsection{Configuration}

Configuration that is specific to a repository is stored in \texttt{.git/config}. It can be edited using \texttt{git config <opt> <val>}.
Global client configuration is stored in \texttt{\string~.gitconfig} and can be edited using \texttt{git config --global <opt> <val>}. Some useful options are:


\begin{mdframed}[style=mystyle]
\settowidth{\MyLen}{\texttt{user.name "Your Email"aa}}
\begin{tabular}{@{}p{\the\MyLen}@{}p{\linewidth-\the\MyLen}@{}}
\verb!user.name "Your Name"! & Required if you want to commit. \\
\verb!user.email "Your Email"!  & Also required. \\
\verb!color.ui auto! & Adds colors to git console output. \\
\verb!core.autocrlf true!  & For windows, handles unix-style vs windows line endings. \\
\end{tabular}
\end{mdframed}

\subsection{Create a repository}
\begin{mdframed}[style=mystyle]
\settowidth{\MyLen}{\texttt{git clone <url> [name]} \ }
\begin{tabular}{@{}p{\the\MyLen}%
                @{}p{\linewidth-\the\MyLen}@{}}
\texttt{git clone <url>} & Copy an entire repository \\
\texttt{git clone <url> [name]} & Copy an entire repository, rename the folder to \texttt{[name]} \\
\texttt{git init} & Create a Git repository in the current folder \\
\texttt{git init [name]} & Create a Git repository in \texttt{[name]}. \texttt{[name]} is created if it doesn't exist. \\
\texttt{git init --bare} & Create a bare repository (for Git servers) \\
\end{tabular}
\end{mdframed}

\subsection{View the state}
\begin{mdframed}[style=mystyle]
\settowidth{\MyLen}{\texttt{git diff --cached} \ }
\begin{tabular}{@{}p{\the\MyLen}%
                @{}p{\linewidth-\the\MyLen}@{}}
\texttt{git status} & List modified, untracked, staged \\
\texttt{git branch} & List all local branches \\
\texttt{git branch -a} & List all local and remote branches \\
\texttt{git diff} & View non-indexed changes (since \texttt{HEAD}) \\
\texttt{git diff --cached} & View indexed changes (since \texttt{HEAD}) \\
\texttt{git remote [-v]} & List all remotes [and their URLs] \\
\texttt{git show [commit]} & Dump changeset introduced in \texttt{[commit]}. Defaults to \texttt{HEAD}. \\
\texttt{git blame [file]} & Dump file contents with annotations of the last commit that modified each line. \\
\texttt{gitk --all \&} & Nice GUI for viewing both local and remote branches \\
\texttt{tig} & Text mode repository browser \\
\end{tabular}
\end{mdframed}

\subsection{Make changes}
\begin{mdframed}[style=mystyle]
\settowidth{\MyLen}{\texttt{git reset -- [file]} }
\begin{tabular}{@{}p{\the\MyLen}%
                @{}p{\linewidth-\the\MyLen}@{}}
\texttt{git add [new-file]} & Add an untracked file to the index. Will be tracked by Git henceforth. \\
\texttt{git add [file]} & Index all changes in \texttt{[file]} \\
\texttt{git add -p} & Index changes selectively \\
\texttt{git add .} & Index all untracked (non-ignored) and modified files in path and subfolders. \\
\texttt{git add -u} & Index all modified files and deletions. Do nothing to untracked content. \\
\texttt{git add -A} & Same as \texttt{git add . \&\& git add -u} \\
\texttt{git reset -- [file]} & Unstage all contents of \texttt{[file]} \\
\texttt{git checkout <ref> -- [file]} & Write to disk the version of \texttt{[file]} from \texttt{<ref>}. \\
\texttt{git rm [file]} & Stage the removal of \texttt{[file]}, and delete it from the working tree. \\
\texttt{git rm --cached  [file]} & Stage the deletion of \texttt{[file]}, but keep it locally as untracked. \\
\texttt{git rm -r [dir]} & Stage the removal of a directory \& contents recursively. \\
\texttt{git commit -m \string"..\string"} & Apply the staged modifications to \texttt{HEAD} \& make a new commit out of it. \\
\texttt{git commit -am \string"..\string"} & Stage all modifications, then commit. \\
\texttt{git revert <ref>} & Undo changes introduced in \texttt{<ref>} by applying a new commit. \\
\end{tabular}
\end{mdframed}

\subsection{Tip: deleting files}
\begin{mdframed}[style=tip]
If you remove a file outside of git (like \texttt{rm [file]}), the file is deleted but git does not index that change. Fix with \texttt{git add -u} if you don't mind indexing your modified files as well. Otherwise, you must call \texttt{git rm [file]} on all deleted files or automatically with:

\vspace{1mm} 

\texttt{git ls-files --deleted -z | xargs -0 git rm}

\vspace{1.5mm} 

To undo \texttt{git rm [file]} is a 2-step process. 

\settowidth{\MyLen}{2. \texttt{git checkout -- <file> }} 
\begin{tabular}{@{}p{\the\MyLen}%
                @{}p{\linewidth-\the\MyLen}@{}}
1. \texttt{git reset -- <file>} & Unstage the file deletion. \\
2. \texttt{git checkout -- <file>} & Replace the file with a copy from the last commit. \\
\end{tabular}
\end{mdframed}

\subsection{Branch Operations}
\begin{mdframed}[style=mystyle]
\settowidth{\MyLen}{\texttt{git checkout <br>.} }
\begin{tabular}{@{}p{\the\MyLen}%
                @{}p{\linewidth-\the\MyLen}@{}}
\texttt{git branch <br> [ref]} & Create a branch called \texttt{<br>}, at \texttt{[ref]}, or at \texttt{HEAD} by default.\\
\texttt{git checkout <br>} & Switch to branch \texttt{<br>} \\
\texttt{git checkout <ref>} & Switch to the commit referenced by the \texttt{<ref>}. Will be in detached head mode if it is not a branch name. \\
\texttt{git reset <ref>} & Move the current branch to \texttt{<ref>}, preserving the modifications (set as unstaged). \\
\texttt{git reset --soft <ref>} & Move the current branch to \texttt{<ref>}, preserving the modifications (set as staged). \\
\texttt{git reset --hard <ref>} & Move the current branch to \texttt{<ref>}, dropping all differences and local modifications. \\
\texttt{git merge <ref>} & Incoporated changes from \texttt{<ref>} into the current branch. \\
\texttt{git merge --no-ff <ref>} & Force a merge commit, even when fast-forward is possible. \\
\texttt{git rebase <ref>} & Apply differences between HEAD and \texttt{<ref>} upon \texttt{<ref>}. \\
\texttt{git rebase <ref> -i} & Interactive rebase for history modifications \\
\end{tabular}
\end{mdframed}

\subsection{Synchronizing with Remotes}
\begin{mdframed}[style=mystyle]
\settowidth{\MyLen}{\texttt{git push <alias> <a>[:b]} }
\begin{tabular}{@{}p{\the\MyLen}%
                @{}p{\linewidth-\the\MyLen}@{}}
\texttt{git fetch <alias>}        & Get update from a remote \\
\texttt{git push <alias> <br> }    & Write branch \texttt{<br>} to remote called \texttt{alias} \\
\texttt{git push <alias> <a>[:b]} & Write branch \texttt{<a>} to remote called \texttt{alias} and name it \texttt{<b>}. \\
\texttt{git push  -f ...}          & Overwrite a branch, if necessary \\
\texttt{git push <alias> --tags}    & Send all tags to a remote\\
\texttt{git pull <alias>}         & Combines a \texttt{git fetch} and \texttt{git merge} or \texttt{git rebase} \\
\texttt{git remote set-url <alias> <new-url>}  & Define a new URL for a remote \\
\texttt{git remote add <alias> <url>} & Add a new remote \\
\texttt{git remote rm <alias>} & Remove a remote \\
\end{tabular}
\end{mdframed}


%\rule{0.3\linewidth}{0.25pt}
%\scriptsize
%
%Copyright \copyright\ 2015 Dawans Engineering, SPRL \href{http://www.dawans.be}{http://www.dawans.be}

\end{multicols}
\end{document}
