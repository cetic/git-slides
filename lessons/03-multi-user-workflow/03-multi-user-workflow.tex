% !Mode:: "TeX:UTF-8"
\documentclass{../common/tufte-latex/tufte-handout}

\title{Git hands-on, part II: multiple user workflows}

\author{S\'ebastien Dawans}

\date{07 February 2014} % without \date command, current date is supplied

%\geometry{showframe} % display margins for debugging page layout
\usepackage[utf8]{inputenc}
\usepackage{graphicx} % allow embedded images
  \setkeys{Gin}{width=\linewidth,totalheight=\textheight,keepaspectratio}
  \graphicspath{{graphics/}} % set of paths to search for images
\usepackage{amsmath}  % extended mathematics
\usepackage{booktabs} % book-quality tables
\usepackage{units}    % non-stacked fractions and better unit spacing
\usepackage{multicol} % multiple column layout facilities
\usepackage{lipsum}   % filler text
\usepackage{fancyvrb} % extended verbatim environments
  \fvset{fontsize=\normalsize}% default font size for fancy-verbatim environments
\usepackage{listings}
\lstset{showstringspaces=false}
\usepackage[usenames]{xcolor}

\lstdefinestyle{BashInputStyle}{
  language=bash,
  basicstyle=\footnotesize\ttfamily,
  %numbers=left,
  %numberstyle=\tiny,
  %numbersep=3pt,
  frame=tb,
  columns=fullflexible,
  backgroundcolor=\color{yellow!20},
  linewidth=0.95\linewidth,
  xleftmargin=0.05\linewidth,
  moredelim=**[is][\color{red}]{§}{§},
  moredelim=**[is][\color{OliveGreen}]{`}{`}
}

% Standardize command font styles and environments
\newcommand{\doccmd}[1]{\texttt{\textbackslash#1}}% command name -- adds backslash automatically
\newcommand{\docopt}[1]{\ensuremath{\langle}\textrm{\textit{#1}}\ensuremath{\rangle}}% optional command argument
\newcommand{\docarg}[1]{\textrm{\textit{#1}}}% (required) command argument
\newcommand{\docenv}[1]{\textsf{#1}}% environment name
\newcommand{\docpkg}[1]{\texttt{#1}}% package name
\newcommand{\doccls}[1]{\texttt{#1}}% document class name
\newcommand{\docclsopt}[1]{\texttt{#1}}% document class option name
\newenvironment{docspec}{\begin{quote}\noindent}{\end{quote}}% command specification environment

\begin{document}

\maketitle% this prints the handout title, author, and date

\begin{abstract}
\noindent
This handout is part of a series of practical courses on Git.
The previous course, \textbf{part I: single user operations}, introduces Git commands in a single-user setting, mainly for practical reasons to avoid users interacting with the same repository without basic knowledge of Git.
In practice, of course, it's often desireable to centralize contributions from several developers into a common, unique repository.
Git doesn't impose a specific workflow for this. 
In this lesson, we will cover the essentials on managing a multi-developer project with Git.
We first cover basic concepts, then introduce a popular workflow, git-flow, as an applied example.
We also cover more advanced concepts, like rewriting history.
\end{abstract}

\section{Preparation}

To prepare for this session, it is a good idea to re-read the previous handout as a refresher.
This session uses the lesson2 repository as central Git repository, hosted on your local Gitlab server.
Make sure you are able to access the lesson2 repository in the web interface, and try a clone:

\begin{lstlisting}[style=BashInputStyle]
  $ git clone git@gitlab.server.com:git-training/lesson2 [folder]
\end{lstlisting}

Remember, \texttt{git clone} creates a new folder, lesson2, with a self-contained git repository (.git folder) and the working tree of the latest version of the default branch, master in this case.

\section{Single remote branch workflows}

In the last lesson, we added features to a local \textbf{master} branch, then pushed this branch regularly to a remote master branch.
Progessing the local master branch was done in different ways.

\subsection{Single, local branch}
The first workflow we covered was adding commits directly to the master branch.
This is the most simple workflow.
\marginnote{Remember to use \textbf{gitk} often, with the --all option, as it's a very convenient way to visualize the relationship between local branches and those of one or more remotes.}
\begin{lstlisting}[style=BashInputStyle]
  $ git clone <repo url>
  $ git branch
  * `master`
  $ git add <new/modified stuff>
  $ git commit -m "Commit Message"
  $ git push origin master
\end{lstlisting}

This is very simple, but doesn't address common questions and issues that arise when tracking large projects and/or when dealing with several developers.

\subsection{Local feature branches}
We then introduced the \textbf{git branch} command, to create local branches, and \textbf{git checkout} to switch the working tree's contents from that of one branch to another.
As a reminder, this is a typical workflow for working in a feature branch, then merging it into the local master branch before contributing:

\begin{lstlisting}[style=BashInputStyle]
  $ git clone <repo url>
  $ git branch
  * `master`
  $ git branch feature-42 
  $ git checkout feature-42
  $ git add <new/modified stuff>
  $ git commit -m "Commit Message"
  $ git checkout master
  $ git merge feature-42 [--no-ff]
  $ git push origin master
\end{lstlisting}

In this context, the feature branch is used locally to isolate modifications until they are merged in the master branch.
Like a lot of short-lived feature branches, it was never actually pushed to the remote repository.

If working on a long-term feature, it may be a good practice to push the feature branch to the remote repository for the sake of not losing your work:

\begin{lstlisting}[style=BashInputStyle]
  $ git push origin feature-42
\end{lstlisting}

This doesn't affect the remote master branch in any way, just like our local feature-42 branch has not affected our local master branch until we merged.
You can always decide to delete it later on:
\marginnote{The general syntax for pushing a branch is git push remotealias localbranch:remotebranch, i.e. you can change the name when pushing. Thus, the weird :remotebranch syntax pushes nothing into remotebranch --> remotebranch is deleted!}
\begin{lstlisting}[style=BashInputStyle]
  $ git push origin :feature-42
\end{lstlisting}

\subsection{Multiple long-term remote branches}
Sometimes it's useful to have multiple long-term branches on a remote repository.
The most common occurrence of this is the use of a \textbf{master} branch to host stable code, and a \textbf{develop} or \textbf{unstable} branch to host the day-to-day work in progress, not necessarily stable or tested.
A public repository with these two branches typically adresses 2 types of people.
The \textbf{develop} branch will interest contributing developers, who will want the bleeding-edge development version to rebase their contributions on it.
The \textbf{master} branch containing only stable code and updated at each release will be useful for the end-users.

\subsection{TODO}

This handout is still a work in progress...

\bibliography{../common/refs}
\bibliographystyle{plainnat}

\end{document}


% \marginnote{}