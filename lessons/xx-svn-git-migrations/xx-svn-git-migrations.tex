% !Mode:: "TeX:UTF-8"
\documentclass{../common/tufte-latex/tufte-handout}

\title{Git Hands-on, part II-b: Migrating from SVN to Git}

\author{S\'ebastien Dawans}

\date{07 February 2014} % without \date command, current date is supplied

%\geometry{showframe} % display margins for debugging page layout
\usepackage[utf8]{inputenc}
\usepackage{graphicx} % allow embedded images
  \setkeys{Gin}{width=\linewidth,totalheight=\textheight,keepaspectratio}
  \graphicspath{{graphics/}} % set of paths to search for images
\usepackage{amsmath}  % extended mathematics
\usepackage{booktabs} % book-quality tables
\usepackage{units}    % non-stacked fractions and better unit spacing
\usepackage{multicol} % multiple column layout facilities
\usepackage{lipsum}   % filler text
\usepackage{fancyvrb} % extended verbatim environments
  \fvset{fontsize=\normalsize}% default font size for fancy-verbatim environments
\usepackage{listings}
\lstset{showstringspaces=false}
\usepackage[usenames]{xcolor}
\usepackage{soul}
\usepackage{enumerate}

\lstdefinestyle{BashInputStyle}{
  language=bash,
  basicstyle=\footnotesize\ttfamily,
  %numbers=left,
  %numberstyle=\tiny,
  %numbersep=3pt,
  frame=tb,
  columns=fullflexible,
  backgroundcolor=\color{yellow!20},
  linewidth=0.95\linewidth,
  xleftmargin=0.05\linewidth,
  moredelim=**[is][\color{red}]{§}{§},
  moredelim=**[is][\color{OliveGreen}]{°}{°}
}

% Standardize command font styles and environments
\newcommand{\doccmd}[1]{\texttt{\textbackslash#1}}% command name -- adds backslash automatically
\newcommand{\docopt}[1]{\ensuremath{\langle}\textrm{\textit{#1}}\ensuremath{\rangle}}% optional command argument
\newcommand{\docarg}[1]{\textrm{\textit{#1}}}% (required) command argument
\newcommand{\docenv}[1]{\textsf{#1}}% environment name
\newcommand{\docpkg}[1]{\texttt{#1}}% package name
\newcommand{\doccls}[1]{\texttt{#1}}% document class name
\newcommand{\docclsopt}[1]{\texttt{#1}}% document class option name
\newenvironment{docspec}{\begin{quote}\noindent}{\end{quote}}% command specification environment

\begin{document}

\maketitle% this prints the handout title, author, and date

\begin{abstract}
\noindent
This handout is part of a series of practical courses on Git.
\marginnote{git-svn is distributed with the standard Unix Git client \url{https://www.kernel.org/pub/software/scm/git/docs/git-svn.html}}
A common issue when getting started with Git is forgetting old habits adopted while using SVN.
A great remedy for this is to stop using SVN altogether; that sometimes means having to migrate existing, long-term projects from SVN to Git.
Here, we will see how easy it is to migrate SVN projects to Git with the \texttt{git-svn} tool.
A migration can take as few as 10 minutes, so let's get started!
\end{abstract}

\section{What this handout is not}

This is \textit{not} a complete reference on how the git-svn tool works.
There is already plenty of documentation on that, a great one being the Pro Git book by Scott Chacon, available online \cite{git-scm-ch8}.
git-svn is a Git client offering features of Git where branching, merging, staging, committing, viewing and rewriting the history can all be done locally.
\marginnote{Rewriting non-pushed, or non "SVN-comitted" history, to be exact.}
Git fanatics stuck on pre-historic SVN-based projects may certainly find this useful.

This handout only scratches the surface of git-svn as an SVN proxy.
Instead, we will focus on leveraging this tool to migrate SVN projects to Git once and for all.

\section{Preparation}

\subsection{Installation}

For this tutorial, we will use a sample SVN project hosted on good ol' SourceForge. 
\marginnote{SourceForge definition of SVN: Enterprise-class centralized version control for the masses. Classy.}
That means you'll need both a Git client and optionally an SVN client.
\marginnote{We will only use SVN for browsing through the project, but the migration doesn't require it.}
The Git installation is already covered in a previous handout.

To install SVN, Linux users should install the \textbf{subversion} package.

\begin{lstlisting}[style=BashInputStyle]
  $ apt-get install subversion
\end{lstlisting}

Windows users may download and execute the TortoiseSVN installer and verify that the svn command is properly added to the \texttt{PATH} variable.
\marginnote{TortoiseSVN \url{http://tortoisesvn.net/downloads.html}}

\subsection{\st{Clone} Checkout the SVN project}

Last but not least, let's execute our one and only SVN command to grab the SVN project from SourceForge.

\begin{lstlisting}[style=BashInputStyle]
  $ svn checkout svn://svn.code.sf.net/p/svngitmigration/code/ svngit-project
  $ cd svngit-project
\end{lstlisting}

Having a local checkout of the SVN project is not a required step for the migration, but we're doing this to have a look at the structure of the SVN project before migrating.

\section{Analyzing the SVN project}

A typical SVN project contains a trunk with the latest development version of the application.
The trunk is an actual physical folder in the working tree, usually named \textbf{trunk}.
As opposed to Git, branches and tags have a physical location in the working tree as well, so a common practice is to store these in top-level \textbf{branches} and \textbf{tags} folders, next to the trunk.
Our sample project follows this convention:

\begin{lstlisting}[style=BashInputStyle]
  $ tree -L 2
  .
  |-- branches
  |   |-- feature1
  |-- tags
  |   |-- r1.0
  |   |-- r1.1
  |-- trunk
    |-- README.txt
    |-- src

\end{lstlisting}

Let's check the log:

\begin{lstlisting}[style=BashInputStyle]
  $ svn log
\end{lstlisting}

Before we go on, let's consider some differences between SVN and Git \footnote{Other than the time it took to display the log of 12 commits} and what this implies in terms of migrating to Git:

\begin{itemize}
 \item \textbf{SVN branches and tags have a physical location}. This is a major difference to Git, where branches and tags are very lightweight: mere labels and pointers to commits. Thus, there isn't a 1-to-1 match between an SVN working tree and a Git working tree.
 \item \textbf{Author names are not formatted in the same way}. SVN doesn't require any structure for author names, it's usually the committer's account username on the Subversion system. On the other hand, Git formats the author information with \texttt{"First Last <email@domain.com>"}. We need to provide a mapping for this.
 \item \textbf{SVN accepts blank commits, Git doesn't}. True, but don't panic. Git will allow the undisciplined SVN user to import his blank commits.
\end{itemize}

\marginnote{You can also visualize it on SourceForge \url{http://sourceforge.net/p/svngitmigration/code}}
We can delete this local checkout, we don't need it.

\begin{lstlisting}[style=BashInputStyle]
  $ cd ..
  $ rm -rf svngit-project
\end{lstlisting}

\section{SVN to Git Migration}

\subsection{Step 1: prepare authors mapping}
We need to create a file which contains the mapping of SVN and Git authors.
The format of this file is 1 line per author:

\begin{lstlisting}[style=BashInputStyle]
  SVN name = Git name
  SVN name = Git name
  ...
\end{lstlisting}

If you're lazy and you don't want to figure out who has been working on the project you can use the following bash script to poll the SVN repository and dump a list of author names:

\begin{lstlisting}[style=BashInputStyle]
authors=$(svn log -q http://svn.code.sf.net/p/svngitmigration/code/ | \ 
           grep -e '^r' | awk 'BEGIN { FS = "|" } ; { print $2 }' | \
           sort | uniq)
           for author in ${authors}; do
             echo "${author} = ${author%@*} <${author}>";
           done
\end{lstlisting}

There is only 1 author in this project, so it will output:

\begin{lstlisting}[style=BashInputStyle]
  sdawans = sdawans <sdawans>
\end{lstlisting}

Now simply copy/paste this into a new file, which we will call 'authors', and adjust the Git side:

\begin{lstlisting}[style=BashInputStyle]
  $ touch authors
  $ echo "sdawans = Sebastien Dawans <sebastien@dawans.be>" >> authors
\end{lstlisting}

\subsection{Step 2: Git SVN Clone}
The magic command which does about 80\% of the work is \textbf{git svn clone <url>}.
We will add a few options to it, however, to make it even smarter.
The options are:

\noindent \textbf{-- no-metadata:} Drops some superfluous info in commit messages which the tool normally adds for the general use case of the git svn tool as an SVN proxy.

\noindent \textbf{-- authors-file:} Path to our authors mapping file.

\noindent \textbf{-- trunk:} Path to the trunk directory from the project root

\noindent \textbf{-- branches:} Path to the branches directory.

\noindent \textbf{-- tags:} Path to the tags directory.

\noindent Let's put it all together:

\begin{lstlisting}[style=BashInputStyle]
  $ git svn clone http://svn.code.sf.net/p/svngitmigration/code/ \
    --authors-file=authors --no-metadata --trunk=trunk \ 
    --branches=branches --tags=tags
\end{lstlisting}

This will create a new Git repository, in a folder named \textbf{code}, which we will analyse now:

\begin{lstlisting}[style=BashInputStyle]
  $ cd code
  $ tree
  .
  |-- README.txt
  |-- src
      |-- calc.py
\end{lstlisting}

The Git and SVN working tree are very different.
The Git working tree contains a \textbf{master} branch, whos version is the latest version of the \textbf{trunk} directory of the SVN project.
Branches and Tags are not part of the working tree itself in Git, and the tool has already taken this into account.

\begin{lstlisting}[style=BashInputStyle]
  $ git branch -a
  * `master`
    §remotes/feature1§
    §remotes/tags/r1.0§
    §remotes/tags/r1.1§
    §remotes/trunk§
\end{lstlisting}

We see that Git has stored remote branches for each feature and each tag defined in the SVN repository.
There is no physical remote behind this, it is only a virtual remote acting as a proxy for the physical branches and tags in the SVN working tree.

\subsection{Step 3: Convert tags to real Git tags}

\begin{lstlisting}[style=BashInputStyle]
  $ git tag 1.0 remotes/tags/r1.0
  $ git tag 1.1 remotes/tags/r1.1
\end{lstlisting}

Or we can use the annotated tag if we prefer to add a meaningful description:

\begin{lstlisting}[style=BashInputStyle]
  $ git tag -a 1.0 remotes/tags/r1.0 -m "Tag 1.0 description"
  $ git tag -a 1.1 remotes/tags/r1.1 -m "Tag 1.1 description"
\end{lstlisting}

\subsection{Step 4: Create local copies of the branches}

This creates a local branch called feature1, using the virtual remotes/feature1 branch: 
\begin{lstlisting}[style=BashInputStyle]
  $ git branch feature1 remotes/feature1
\end{lstlisting}

It should be repeated for each branch when there are several.

\subsection{Step 5: Push to a Git Repo}

The final step is to push to your new Git repository.
Assuming you have already created a project on a Git server on which you have RW access,

\begin{lstlisting}[style=BashInputStyle]
  $ git remote add origin <url>
  $ git push origin --all --tags
\end{lstlisting}

This pushes all branches and all tags.

\section{More automation}

Steps 3 and 4 can be automated in a single command.

To tag all tags at once:
\begin{lstlisting}[style=BashInputStyle]
  $ git for-each-ref refs/remotes/tags | cut -d / -f 4- | \
    grep -v @ | while read tagname; \
    do git tag "$tagname" "tags/$tagname"; \
    git branch -r -d "tags/$tagname"; done
\end{lstlisting}

To create all branches in 1 command:
\begin{lstlisting}[style=BashInputStyle]
  $ git for-each-ref refs/remotes | cut -d / -f 3- | \
    grep -v @ | while read branchname; \
    do git branch "$branchname" "refs/remotes/$branchname"; \
    git branch -r -d "$branchname"; done
\end{lstlisting}

\section{Manual Optimizations}

The automated procedures given above are well suited for well organized projects, or simple ones with only a trunk.
However, an SVN repo isn't always formatted the way you would have like it to be.
It is not rare to encounter repositories with large binary files, libraries and generated files which have been version controlled by mistake or lack of discipline.
An SVN to Git migration is a good opportunity to clean up some of these issues.

The best time to unversion unwanted files is \textbf{during} the migration to git itself to avoid adding large blobs to the git object base.
The \textbf{git svn clone} command accepts an \textbf{--ignore-path} option formatted using a Perl regex.

This example excludes all folders called bin, doc and lib within the entire SVN project:

\begin{lstlisting}[style=BashInputStyle]
  $ git svn --authors-file=authors clone <url> \ 
    --no-metadata --ignore-paths="^bin|^doc|^lib"
\end{lstlisting}

Let's remember to add a .gitignore file while we are at it, to keep ignoring these in Git:

\begin{lstlisting}[style=BashInputStyle]
  $ printf "bin\nlib\ndoc\n" > .gitignore
\end{lstlisting}


\bibliography{../common/refs}
\bibliographystyle{plainnat}

\end{document}
